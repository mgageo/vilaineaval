% <!-- coding: utf-8 -->
\twocolumn
\renewcommand{\monlhead}{Enquête oiseaux en hiver}
\subsection*{Contexte}
Rennes Métropole prévoit d'aménager la vallée de la Vilaine entre Rennes et Laillé. Le patrimoine naturel de ce secteur n'est que très partiellement connu. En oiseaux, à part quelques sites emblématiques, nous ne disposons quasiment pas de données.


Nous lançons sur le secteur Nord (La Prévalaye, Lillion, La Piblais)  une première enquête sur les oiseaux en hiver.
Outre le recueil de données ornitho, cette enquête va permettre de sillonner ce vaste secteur de manière à ne rien omettre sur les zones sensibles. Les porteurs du projet oiseaux sont Matthieu Beaufils et Marc Gauthier.
Le projet oiseaux comportent deux volets :
\begin{itemize}
\item une enquête protocolée type SHOM
\item la collecte des données opportunistes
\end{itemize}
\subsection*{Protocole type SHOM}
Le protocole pour cette enquête est inspiré du \href{http://vigienature.mnhn.fr/sites/vigienature.mnhn.fr/files/uploads/Protocole_SHOC_2014_0.pdf}{SHOC} (Suivi Hivernal des Oiseaux Communs) du MNHN.

Le secteur a été découpé en 62 mailles carrées de 0,5 km de côté.

Le protocole HIVER consiste à faire 3 parcours de 5 minutes de marche à petite vitesse (c’est-à-dire 3 parcours compris entre 250 et 350 m) dans CHAQUE maille.

Les parcours seront effectués en fonction des possibilités d’accès sur zone en privilégiant la diversité des milieux.

Pour certaines mailles, il ne sera pas possible de placer trois parcours : surface en eau, zones privées, zone peu pénétrable ...
Dans ces cas il faudra faire :
\begin{itemize}
\item un parcours légèrement à l'extérieur de la maille
\item plusieurs fois le même parcours
\item un point d'écoute
\end{itemize}
En hiver les espèces se déplacent souvent, il faudra en tenir compte lors du cumul des oiseaux posés en fin de parcours.

En résumé les choix faits sont :
\begin{itemize}
\item comptage visuel et auditif de toutes les espèces contactées
\item que les oiseaux posés
\item pas les oiseaux sur les plans d'eau
\item uniquement les nombres d'oiseaux, pas de distance ou de localisation
\item comptage entre 9h30 et 14h00
\item période 1er décembre au 20 janvier
\end{itemize}
\subsubsection*{Autres informations}
Il faudra profiter de ces parcours pour noter quelques informations sur le milieu :
\begin{itemize}
\item boisement
\item zones humides, mares
\item météo
\item activités humaines : pêche, jogging, rencontres, crassiers
\end{itemize}
\subsubsection*{Prospection}
Le document \href{https://docs.google.com/spreadsheets/d/10uaf6o8Xu8dBiizxhHWJxUXgdR-Iu6mg9Utc5pJgUBY/edit?usp=sharing}{Maille Obseur} permet de suivre l'avancement des prospections.
Avant de partir prospecter une maille, il faut indiquer son nom dans la ligne correspondante.
Au retour, il faut indiquer le nombre de parcours réalisés et la date.

Le recueil des informations collectées sera fait sur :
\begin{itemize}
\item une carte, pour le tracé des parcours
\item un bordereau, pour les données
\end{itemize}

Le document pdf \href{https://drive.google.com/open?id=0BzQEzw-5wGqta0pvZDRxVmtDZHM}{cartes.pdf} fournit par maille une page avec :
\begin{itemize}
\item une carte de situation
\item une vue aérienne
\end{itemize}
Cette carte permettra de tracer les trois parcours : A, B et C.

La feuille de calcul \href{https://drive.google.com/open?id=1TipmUPvPvoTlDldVDKbu0oJvzKWfrje2e-gMe-u-XCI}{Bordereau données} permettra de saisir les informations pour les troix parcours.


\subsubsection*{Saisie des données protocolées}
Cette saisie sera effectuée :
\begin{itemize}
\item données : sur Serena par Matthieu
\item parcours : sur Qgis par Marc
\end{itemize}
Il faudra donc renvoyer la carte et le bordereau à rennes@bretagne-vivante.org.

Au fur et à mesure de la saisie, les parcours effectués seront visibles sur cette page \href{http://bretagne-vivante-dev.org/bvo35rva/parcours.html}{web}.

\subsection*{Collecte des données opportunistes}
Cette collecte peut s'effectuer lors d'opérations protocolées ou lors d'un simple passage sur une zone. Elle concerne :
\begin{itemize}
\item les plans d'eau
\item les espèces patrimoniales
\end{itemize}
\subsubsection*{Plans d'eau}
L'enquête Wetlands permet d'évaluer les oiseaux présents (comptage mi-janvier).

Le comptage des différents plans d'eau permettra de conforter les résultats de ce comptage.

\subsubsection*{Espèces patrimoniales}
Pour les données (hors protocole SHOM), par exemple avant ou après les parcours, elles seront sytématiquement notées pour les espèces suivantes :

Alouette des champs
Bergeronnette des ruisseaux
Bouscarle de Cetti
Bouvreuil pivoine
Bruant des roseaux
Bruant jaune
Bruant zizi
Chouettes …
Corbeau freux
Épervier d'Europe
Faucon crécerelle
Fauvette à tête noire
Grimpereau des jardins
Grosbec cassenoyaux
Linotte mélodieuse
Martin-pêcheur
Mésange nonnette
Moineau domestique
Pic épeichette
Pigeon colombin
Pipit farlouse
Pipit spioncelle
Râle d'eau
Roitelets…
Tarier pâtre
Tarin des aulnes


\subsubsection*{Saisie des données opportunistes}
La saisie sera effectuée directement par l'observateur sur faune-bretagne.
Elle s'effectuera en "localisation précise", sans utilisation des formulaires.

En "Remarques", il faudra impérativement saisir "BVO35RVA" en début de champ.



\subsection*{Cartographie web}
Il est très fortement conseillé de bien repérer les parcours potentiels avant de partir sur le terrain.

La page web \href{http://bretagne-vivante-dev.org/bvo35rva}{bvo35rva} permet de visualiser :
\begin{itemize}
\item la grille
\item l'inventaire communal des zones humides du SAGE Vilaine
\item la localisation des milieux naturels d'intérêt écologique (MNIE)
\item les habitats des MNIE
\end{itemize}
Un clic sur la carte permet d'obtenir les informations sur ces zones.



